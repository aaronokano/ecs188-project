\documentclass[12pt]{article}
\usepackage{amsmath, amsthm, amssymb}
\usepackage{listings}
\usepackage{graphicx}
\usepackage{float}
\usepackage{enumerate}
\usepackage{fancyhdr}
\usepackage[labelfont=bf]{caption}
\usepackage[left=0.75in, top=1in, right=0.75in, bottom=1in]{geometry}
\pagestyle{plain}
\begin{document}
\rhead{Aaron Okano\\
       ECS 188, Winter 2013\\
       Presentation}
\thispagestyle{fancy}
\vspace*{0.25in}

% Start writing here

\begin{description}

  \item[Introduction:] ``Since you have all read the paper and are now familiar
    with the ethical issues plaguing the scientific community,lets look at a
    couple of cases where these issues coalesced and had a major impact on
    scientific progress, public spending, and public support of science. In the
    book, \emph{Science, Money, and Politics}, Daniel Greenberg brings up two
    cases, which we will be looking at closely today. These are the Strategic
    Defense Initiative, or SDI, and the Superconducting Supercollider, or SSC.
    Both are megaprojects which raked in billions of dollars of funding with
    very little scientific reasoning for doing so. The reasons the projects
    were able to garner so much financial support intertwines with the ethical
    issues presented in the paper.''

  \item[SDI:] Your part
    \begin{itemize}

      \item``The SDI was proposed by the Reagan administration. Its chief
        goal was to produce a nationwide shield to defend against a nuclear strike
        from the Soviet Union. This proposed shield was to be built using advanced
        technology, consisting largely of relatively undeveloped ideas and relying
        heavily on nuclear power. The core of the plan was to produce targeting
        systems with the ability to destroy an intercontinental ballistic
        missile\ldots with lasers\ldots from space. Now, if that seems overly
        fantastic to you, you are not alone: the media propogated the program's
        nickname of ``Star Wars,'' based on the similarities to technology used in
        the, at the time, recently released film. This, of course, conjures the
        image of the famous Death Star from the first film, [pause] but in reality
        it was far more similar to the half-completed, ultimately-destroyed version
        from the third.''

      \item ``So who do we have to blame for this mess? This boils down to the
        heavy seclusion of the scientific community from national politics and
        the relative influence of those who dare cross this border. One
        particular scientist took this role, and he was not new to the politics
        of weaponry. Edward Teller  % INFORMATION ABOUT TELLER'S HISTORY
        Teller was tapped to join Reagan's White House Science Council. Teller
        developed close ties to Reagan and had the most influence over the
        president out of all the members of the Council. Teller decided to use
        his position to expand the role of the weapons laboratory he co-founded
        in 1952: Lawrence Livermore National Laboratory.

      \item At the time, LLNL was developing the X-ray laser. Teller claimed
        that this laser wolud be the key to a missile defense system. %BROMLEY 284?
        The highly classified nature of the project meant that the
        extraordinary claims boasted by the supporters of the X-ray laser went
        unchecked. %
    \end{itemize}

\end{description}

\end{document}
