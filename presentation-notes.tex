\documentclass[12pt]{article}
\usepackage{amsmath, amsthm, amssymb}
\usepackage{listings}
\usepackage{graphicx}
\usepackage{float}
\usepackage{enumerate}
\usepackage{fancyhdr}
\usepackage[labelfont=bf]{caption}
\usepackage[left=0.75in, top=1in, right=0.75in, bottom=1in]{geometry}
\pagestyle{plain}
\begin{document}
\rhead{Aaron Okano\\
       ECS 188, Winter 2013\\
       Presentation}
\thispagestyle{fancy}
\vspace*{0.25in}

% Start writing here

\begin{description}

  \item[Introduction:] ``Since you have all read the paper and are now familiar
    with the ethical issues plaguing the scientific community,lets look at a
    couple of cases where these issues coalesced and had a major impact on
    scientific progress, public spending, and public support of science. In the
    book, \emph{Science, Money, and Politics}, Daniel Greenberg brings up two
    cases, which we will be looking at closely today. These are the Strategic
    Defense Initiative, or SDI, and the Superconducting Supercollider, or SSC.
    Both are megaprojects which raked in billions of dollars of funding with
    very little scientific reasoning for doing so. The reasons the projects
    were able to garner so much financial support intertwines with the ethical
    issues presented in the paper.''

  \item[SDI]:
    \begin{description}

      \item[What it is:] ``The SDI was proposed by the Reagan administration.
        Its chief goal was to produce a nationwide shield to defend against a
        nuclear strike from the Soviet Union. This proposed shield was to be
        built using advanced technology, consisting largely of relatively
        undeveloped ideas and relying heavily on nuclear power. The core of the
        plan was to produce targeting systems with the ability to destroy an
        intercontinental ballistic missile\ldots with lasers\ldots from space.
        Now, if that seems overly fantastic to you, you are not alone: the
        media propagated the program's nickname of ``Star Wars,'' based on the
        similarities to technology used in the, at the time, recently released
        film. This relation, of course, conjures the image of the famous Death
        Star from the first film, [pause] but in reality it had a far greater
        resemblance to the half-completed, ultimately-destroyed version from
        the third.''

      \item[Teller:] ``So who do we have to blame for this mess? This boils
        down to the heavy seclusion of the scientific community from national
        politics and the relative influence of those who dare cross this
        border. One particular scientist took this role, and he was not new to
        the politics of weaponry. Edward Teller was a leading nuclear
        physicist, and contributor to the Manhattan Project. Dissatisfied with
        the power of the measly fission bomb, Teller sought out nuclear fusion
        and became the ``Father of the Hydrogen Bomb.'' Still yearning to
        create ever more powerful weapons, Teller left Los Alamos labs and
        pushed for the creation of a new weapons research lab---the Lawrence
        Livermore National Laboratory, founded in 1952. Over the next several
        decades, Teller became the foremost promoter of nuclear weapons for
        creative purposes, such as building canals and even shooting down other
        nuclear missles. The latter purpose proved to be appealing to Ronald
        Reagan and Teller was tapped to join his White House Science Council.
        Teller had previously become a close friend to Reagan and had the most
        influence over the president out of all the members of the Council.
        Teller decided to use this position to expand the role of his Lawrence
        Livermore National Laboratory---this time for defense rather than
        offense.

      \item[X-ray laser:] At the time, LLNL was developing the X-ray laser.
        Teller claimed that this laser would be the key to a missile defense
        system. In order to produce a laser powerful enough to destroy a
        nuclear missle mid-flight, Teller proposed discharging a nuclear weapon
        and focusing the X-rays it produces with several lasing crystals, a
        method also discovered at LLNL. This application of the lasers,
        however, was severely overinflated. Several scientists who had worked
        on the X-ray laser even expressed doubt, but not until much later.
        Greenberg interviewed D.\ Allan Bromley, the presidential science
        advisor at the time of SDI's creation and the mentor of a graduate
        student who built the laser, and quotes him as saying, ``The X-ray
        laser was hyped as far as I know from the very beginning\ldots That was
        one of those areas where Ed Teller really hoped to make Livermore a key
        player.'' Reagan's previous science advisor, who personally worked on
        the X-ray laser, commented that ``You know, Los Alamos, with all its
        imperfections, Los Alamos doesn't lie; Livermore lies.'' Of course,
        neither of these advisors had either the influence or the guts to stand
        against Teller and the president.

      \item[Science community:] What was the rest of the scientific community
        doing during this time? The project itself was no secret to the public,
        and while some of the details were classified, it was a well-known fact
        that the system revolved around X-ray lasers. Well, many of the
        scientists were jumping at the prospect of more money. In fact, the
        Strategic Defense Initiative Organization reported that over 3,000
        university scientists applied for funds to aid in the missile defense
        endeavor. The highly classified nature of the project meant that the
        extraordinary claims boasted by the supporters of the X-ray laser went
        unchecked.

    \end{description}

\end{description}

\end{document}
