\documentclass[12pt]{article}
\usepackage{amsmath, amsthm, amssymb}
\usepackage{listings}
\usepackage{graphicx}
\usepackage{float}
\usepackage{enumerate}
\usepackage{fancyhdr}
\usepackage[labelfont=bf]{caption}
\usepackage[left=0.75in, top=1in, right=0.75in, bottom=1in]{geometry}
\pagestyle{plain}
\begin{document}
\rhead{Aaron Okano\\
       ECS 188, Winter 2013\\
       Presentation}
\thispagestyle{fancy}
\vspace*{0.25in}

% Start writing here

\begin{description}

  \item[Introduction:] ``Hello and good afternoon! Today Jason and I will be
    talking about megaprojects---specifically, US government sponsored
    megaprojects, and the ethical issues that can arise from an intimate
    intertwining of scientific and political goals which are often present in
    large undertakings like the ones we will be looking at today. Since you
    have all read the paper and are now familiar with the general ethical
    issues plaguing the scientific community, we will take a close look at a
    couple of cases where these issues coalesced and had a major, negative
    impact society. In the book, \emph{Science, Money, and Politics}, Daniel
    Greenberg has two good examples of this. These are the Strategic Defense
    Initiative, or SDI, and the Superconducting Supercollider, or SSC. Both
    are megaprojects which raked in billions of dollars of funding with very
    little scientific reasoning for doing so. The reasons the projects were
    able to garner so much financial support intertwines with the ethical
    issues presented in the paper.''

  \item[SDI]:
    \begin{description}

      \item[What it is]: 
        \begin{itemize}

          \item It was a defense program with the goal of producing a network
            of anti-ballistic missile defenses to shield North America from
            nuclear attack by the Soviet Union.

          \item It was proposed by President Ronald Reagan in 1983 and began
            development in 1984.

          \item The shield was to be built using advanced technology,
            consisting of relatively undeveloped technologies and heavily based
            in nuclear power.

          \item The core of the program was invested in the idea that the
            targeting systems would be able to destroy intercontinental
            ballistic missiles\ldots with lasers\ldots from space.

          \item Now, if that seems overly fantastic to you, you are not alone:
            the media branded the program as ``Star Wars,'' referencing the
            recently released film.

          \item This relation conjures the image of the famous Death Star from
            the first film in the original trilogy.

          \item (pause) The reality is that it had a far greater resemblance to
            the Death Star from the third movie.
        
          %\item ``The SDI was proposed by the Reagan administration.  Its chief
          %  goal was to produce a nationwide shield to defend against a nuclear
          %  strike from the Soviet Union. This proposed shield was to be built
          %  using advanced technology, consisting largely of relatively
          %  undeveloped ideas and relying heavily on nuclear power. The core of
          %  the plan was to produce targeting systems with the ability to
          %  destroy an intercontinental ballistic missile\ldots with
          %  lasers\ldots from space.  Now, if that seems overly fantastic to
          %  you, you are not alone: the media propagated the program's nickname
          %  of ``Star Wars,'' based on the similarities to technology used in
          %  the, at the time, recently released film. This relation, of course,
          %  conjures the image of the famous Death Star from the first film,
          %  [pause] but in reality it had a far greater resemblance to the
          %  half-completed, ultimately-destroyed version from the third.''

        \end{itemize}

      \item[Teller]:
        \begin{itemize}

          \item So where did this project come from? It boils down to two
            issues: the seclusion of the scientific community from national
            politics and the relative influence of those scientists who dare
            enter the public arena.

          \item One scientist in particular took on the latter role, and he was
            no newcomer to the politics of weaponry.

          \item Edward Teller was a leading nuclear physicist

          \item Contributer to the Manhattan Project

          \item Shifted to fusion research when fission created an inadequate
            bomb for him

          \item Left Los Alamos National Labs to push for formation of Lawrence
            Livermore National Laboratories, which was founded in 1952

          \item Pursued further research of nuclear weapons and avidly promoted
            their use

          \item Thought up creative ways to use nukes, such as to dig out
            canals and to shoot down nuclear missiles

          \item Reagan liked the latter idea and appointed Teller to his White
            House Science Council

          \item Teller was a close friend to Reagan and had most influence in
            the Council

          \item Teller used this position to promote both LLNL and his nuclear
            missile defense plans

        \end{itemize}
        
        
        %``So who do we have to blame for this mess? This boils
        %down to the heavy seclusion of the scientific community from national
        %politics and the relative influence of those who dare cross this
        %border. One particular scientist took this role, and he was not new to
        %the politics of weaponry. Edward Teller was a leading nuclear
        %physicist, and contributor to the Manhattan Project. Dissatisfied with
        %the power of the measly fission bomb, Teller sought out nuclear fusion
        %and became the ``Father of the Hydrogen Bomb.'' Still yearning to
        %create ever more powerful weapons, Teller left Los Alamos labs and
        %pushed for the creation of a new weapons research lab---the Lawrence
        %Livermore National Laboratory, founded in 1952. Over the next several
        %decades, Teller became the foremost promoter of nuclear weapons for
        %creative purposes, such as building canals and even shooting down other
        %nuclear missles. The latter purpose proved to be appealing to Ronald
        %Reagan and Teller was tapped to join his White House Science Council.
        %Teller had previously become a close friend to Reagan and had the most
        %influence over the president out of all the members of the Council.
        %Teller decided to use this position to expand the role of his Lawrence
        %Livermore National Laboratory---this time for defense rather than
        %offense.

      \item[X-ray laser]:
        \begin{itemize}

          \item At this time, LLNL was developing a new kind of laser---the
            X-ray laser

          \item Teller believed that this type laser could be used for missile
            defense if one were to use the X-rays produced through nuclear
            fission and focusing them with a dozen lasing crystals to strike a
            dozen targets at once. This method was also discovered at LLNL

          \item That claim, however, was severely overinflated

          \item Several scientists who had worked on the lasers or had close
            relations with scientists who developed the laser were even
            skeptical

          \item Greenberg interviewed two of these scientists

          \item One was D.\ Allan Bromley, the presidential science advisor at
            the time SDI was announced and the mentor of a graduate student who
            built the first X-ray lasers. Greenberg quotes him as saying, ``The
            X-ray laser was hyped as far as I know from the very
            beginning\ldots That was one of those areas where Ed Teller really
            hoped to make Livermore a key player.''

          \item Reagan's previous science advisor, who personally worked on the
            development of the laser, also had concerns about Teller's claims.
            When asked about it, he told Greenberg, ``You know, Los Alamos,
            with all its imperfections, Los Alamos doesn't lie; Livermore
            lies.''

          \item Unfortunately, neither of these advisors had either the
            influence or the guts to stand against Teller and the president

        \end{itemize}
        
        %At the time, LLNL was developing the X-ray laser.
        %Teller claimed that this laser would be the key to a missile defense
        %system. In order to produce a laser powerful enough to destroy a
        %nuclear missle mid-flight, Teller proposed discharging a nuclear weapon
        %and focusing the X-rays it produces with several lasing crystals, a
        %method also discovered at LLNL. This application of the lasers,
        %however, was severely overinflated. Several scientists who had worked
        %on the X-ray laser even expressed doubt, but not until much later.
        %Greenberg interviewed D.\ Allan Bromley, the presidential science
        %advisor at the time of SDI's creation and the mentor of a graduate
        %student who built the laser, and quotes him as saying, ``The X-ray
        %laser was hyped as far as I know from the very beginning\ldots That was
        %one of those areas where Ed Teller really hoped to make Livermore a key
        %player.'' Reagan's previous science advisor, who personally worked on
        %the X-ray laser, commented that ``You know, Los Alamos, with all its
        %imperfections, Los Alamos doesn't lie; Livermore lies.'' Of course,
        %neither of these advisors had either the influence or the guts to stand
        %against Teller and the president.


      \item[Science community:] What was the rest of the scientific community
        doing during this time? The project itself was no secret to the public,
        and while some of the details were classified, it was a well-known fact
        that the system revolved around X-ray lasers. Well, many of the
        scientists were jumping at the prospect of more money. In fact, the
        Strategic Defense Initiative Organization reported that over 3,000
        university scientists applied for funds to aid in the missile defense
        endeavor. However, not all scientists decided to abandon their morals.
        About 2,300 researchers pledged to not apply for or accept funds from
        the Strategic Missile Defense Organization. In the highly divided
        nature of scientists from real politics, such an action is virtually
        meaningless and easily looked past in the national (theatre?). And this
        was the best the scientific community could muster! They rolled over
        and watched as precious funds were diverted away from more productive
        research and sifted into a fairy tale defense program. In fact, from
        its inception in 1984 until 2000, Star Wars outspent cancer research
        three to one. The National Academy of Sciences---the most politically
        influential scientific organization---put out one feeble attempt to
        verify the credibility of the program. Unfortunately, the Academy
        viewed it as necessary for a study for the Academy to have access to
        the classified material surrounding the project; and that access only
        comes through an official request for a study from the White House. To
        put a nail in the coffin, a majority of the Academy members signed a
        petition of opposition to the SDI, which further removed them,
        politically, from being sought after for independent study. It would
        have been possible for the Academy to perform a study using the vast
        collection of unclassified material---specifically, they could have
        looked into the claims about the X-ray lasers---but since there was no
        financial gain for the Academy, they did not even consider that option.

        % This next part references the UNU research
      \item[Budgetary issues:] So the program continued. Naturally, due to the
        sheer size of the project, it immediately began to strain the budget.
        The program began with a budget of \$1 billion in 1984 and quickly grew
        to \$3.9 billion by 1988, with further increases to follow. Over the
        next 15 years, missile defense would come to amass over \$60 billion.
        In 1985, as Congress evaluated the budget needs for Star Wars, the
        Congressional Office of Technology Assesment (OTA) was called into
        action to report on the program. 

    \end{description}

\end{description}

\end{document}
