\documentclass{article}[12pt]
\usepackage{amsmath, amsthm, amssymb}
\usepackage{setspace}
\usepackage{graphicx}
\usepackage{float}
\usepackage{enumerate}
\usepackage{endnotes}
\usepackage[para]{footmisc}
\usepackage{fancyhdr}
\usepackage[labelfont=bf]{caption}
\usepackage[left=1in, top=1in, right=1in, bottom=1.25in]{geometry}


\title{The Abandonment of the Scientific Ethos}  % Declares the document's title.
\author{Aaron Okano, Jason Wong\\Word Count: 2043 }    % Declares the author's name.
\date{March 6, 2013}   % Deleting this command produces today's date.

\begin{document}           % End of preamble and beginning of text.

\maketitle                 % Produces the title.

The scientific research community has long held the outward appearance of
objectivism and progressivism. The public sees little of the true conservative
and political nature of the scientific enterprise, which is willing to abandon
its highly esteemed scientific principles and replace them with the pursuit
of monetary gain and the achievement of public glory. In the latter half of the
twentieth century, scientists have been forgetting their responsibility to
truth and public well-being in favor of their own personal gains. Through his
forty years as a scientific journalist, Daniel Greenberg tells the troubled
story of the modern American scientific enterprise in his book \emph{Science,
Money, and Politics}.

First, it is important to know the players in the politics of science. The
majority of the scientific population abstains from direct involvement in
national politics (aside from casting votes) but still maintains a vital role
in the politics within the science community. For this paper, references to
scientists will be referring to scientists engaged in basic research---research
without direct commercial application---in the so-called ``hard sciences,'' as
opposed to the social sciences. The border between the science community and
Washington is upheld by several organizations. Congress funds two organizations
which in turn fund academic science: the National Science Foundation (NSF) and
the National Institutes of Health (NIH). The White House also has its own
science advisor in some capacity (the position has changed in status several
times over the years) who is in regular communication with scientists. There
also exists the National Academy of Sciences, which is not a part of any
government agency, but still plays a role in the politics of science.

Apart from agencies such as the NSF and the NIH, the scientific community
resides in a realm almost entirely apart from national politics. It has its own
political structure which is built to reward those who have high standing and
political favor amongst scientists. At the center of the scientific political
scene lies the National Academy of Sciences (NAS).  Membership into the society
is highly exclusive, and to be admitted is the most prestigious honor bestowed
upon a scientist, short of the Nobel Prize (p. 262). However, membership is not
always determined by excellence. Carl Sagan, renowned amongst the public for
making science accessible through his television show \emph{Cosmos}, was a
strong contributor to the scientific community with 350 papers (p. 262).  He
also had numerous appointments to NASA advisory positions and played a role in
the deployment of the Mariner, Viking, and Voyager satellites. He was seen by
many to be a natural choice for membership into the academy (p. 262--263). One
obstacle stood in the way of his ascension into glory: he was politically very
different from the existing members. Unlike most accomplished scientists, Sagan
held some status as a celebrity. This led to some resentment within the
scientific community, hence eliminating Sagan not only from contention for the
prize while he was alive, but also from recognition following his death (p.
263). Thus, in the glory-seeking world of science, Sagan's example of
scientific excellence combined with public service was not to be mimicked.

In addition to individual biases within the scientific community, favor towards
certain institutions is also prevalent.  From 1980 to 1995 five universities
have consistently ranked in the top ten recipients of government funding (p.
38).  These top ten receive 17.9\% of all federal R\&D money for universities.
Furthermore, the elite of academia resist changes to the system which would
assist lesser universities in expanding their research. In 1978, the National
Science Foundation launched the Experimental Program for the Stimulation of
Competitive Research (EPSCoR) with the goal of increasing ``the ability of
scientists in eligible states to compete successfully for Federal funds.'' The
academic elite resented this program, and its funding has remained limited
since its inception (p. 39,101).  Over time, America's scientific institutions
have established a system where those wealthy in funds and prestige are the
primary recipients of funding.  Unfortunately, this system is firmly in place
and unlikely to budge. As Reagan's White House science advisor remarked,
``preserving the status quo has become the overarching goal, replacing the
pursuit of excellence.'' (p. 34)

This structure that creates an aristocracy of academics has in turn grounded
the scientific enterprise in conservatism. As a result, the actions of many
scientists have run contrary to the goals of science. One clear example comes
from an attempt to remedy shortfalls in peer review. In order to improve the
``obtuse and creaky'' peer review system at the National Institutes of Health,
in 1999, the NIH assembled a panel of experts to propose changes which could
help the system. However, afraid that the changes would limit funding, many
university chemists viciously opposed the findings of the panel and the change
was never made (p. 25).

The scientists who are engaged in basic research are contemptuous towards other
fields which they find to be of lesser status. One field in particular has been
the target of science's superiority complex: engineering. Following the
collapse of the Soviet Union, scientists began to question whether the nation
would continue to pour money into basic research since one major political
reason for doing so had been eliminated. Thus, in the 1992 election campaign,
when candidates Bill Clinton and Al Gore stated that ``The absence of a
coherent technology policy is one of the key reasons why America is trailing
some of its major competitors in translating its strength in basic research
into commercial success'' (p. 375), the scientific research community
feared that funding for technology would overrun funding for basic research. 

To address the potentially altered role of the NSF in the future, the
foundation formed a commission to plan for its next five years. Prior to the
commission's creation, the NSF director and the Senate Appropriations
Subcommittee for NSF, more conscious of national politics, suggested that more
emphasis should be placed on applied research (p. 375--377) The commission,
then, contained four members of industry and one politician out of the fifteen
total members---a high number for the academia-dominated NSF (p. 377). As
deliberations commenced, suggestions came flooding in from science and
industry---each biased toward their own self interests (p. 386). Tensions
within the commission were high, and its report was released with so many
muddled opinions that no single, clear message could be drawn from it (p.
388--389). In order to remedy this problem, NSF staff constructed a preface to
the report, and---after several rejected drafts---released what Greenberg calls
an ``artfully composed two-page manifesto'' titled ``In Support of Basic
Research.'' (p. 399). The document contained exactly zero references to
engineering (p. 401). As a further slight to engineers, a 1996 attempt to
change the name of the National Science Foundation to the National Science and
Engineering Foundation lost in the House of Representatives---due to prodding
from the scientific community---with a vote of 339--58 (p. 34--35).

The primary reason that the scientific community perceives its role as being
the top of a hierarchy of knowledge-gatherers is that it clutches to the widely
discredited ``linear model'' of development---that science leads to technology,
but not vice-versa (p. 45). This view was championed by Vannevar Bush, whom
many in math, physics, and chemistry---the dominant voices in science
politics---believe to be the creator of the modern American science
establishment. While a seemingly innocuous belief at first, its danger lies in
the fact that it is used as the primary reason to press for more funding for
science. During the mid-1990s, as the Republican Revolution and promises of
budget cuts arose from Congress, scientists cleverly pointed to recent
technological successes as attributable to advanced scientific knowledge,
despite the fact that some of the most famous of these innovations came from
college dropouts (e.g. Bill Gates and Steve Jobs) (p. 73). The linear model
went unquestioned and the federal budget for science grew to be larger than
ever before (p. 488).

The relationship between science and government has been mutually
beneficial---the scientists get endlessly increasing budgets while the
Congressional politicians use support of science as a political tool. Left out
of this relationship is the general public's interests, which only come as an
occasional side effect of scientific progress. A good example of this is Space
Station Freedom, backed by NASA under the Reagan Administration.  NASA
successfully built political and public support for this endeavor, but the lack
of progress and repetitiveness of its missions took a toll on its supporters.
With the lack of support from the public, NASA's publicists decided to attract
more interest in the space station by creating a competition to send a
schoolteacher to space.  The winner was Christa McAuliffe from New Hampshire.
McAuliffe boarded the space shuttle Challenger, which launched on January 28,
1986 and exploded mid-flight.  This accident was a clear indication that space
exploration was not ready for the public use, yet NASA still followed through
with exploiting the public via a publicty stunt that sent a school teacher to
her death for the sole reason of obtaining financial and public support (p.
411).

University research is funded largely by the federal government, which provided
\$15.5 billion in 1998 (p. 81). With so much money involved, some degree of
corruption is expected, and science is no exception.  Nowhere is this more
prevalent than in the use of the ``indirect cost'' system.  Because the process
of research can have many unexpected costs, grants typically allocate money
towards anything that an university deems is necessary to perform research.
These overhead costs include administration, library facilities, utilities,
depreciation of research equipment and buildings, and student services.
Although all these actions are necessary in order to produce research,
universities stretch the definition of indirect costs to include needless
expenses such as administrative reorganization and hosting official functions
or other social events (p. 85--86).  As time progressed, these indirect costs
increased and began to take funds away from actual research money.  In 1982,
James B. Wyngaarden became the director of the NIH.  He tried to change the
indirect costs system but was only able to create minor adjustments, leaving
the system relatively unchanged.  In 1972, every dollar of direct research
spending had a 30 cent overhead,  By 1990, the overhead had increased to 46
cents per dollar (p. 82).  

Not all money that scientists acquire comes from grants. Through extensive
lobbying, universities have become accomplished in obtaining federal earmarks
as well. It is estimated that lobbying from universities has grossed \$5.1
billion in earmarks between the years of 1980 to 1996 (p. 184). Compared to the
budget for other research-funding organizations, this seems like a trifling
amount. However, earmarks bypass one important aspect of the usual scientific
grant process: there is no peer review. With the lack of peer review comes a
corresponding decline in quality; in fact, it is often the case that earmarks
are sought because of a rejection of a proposal from the reviewers (p.
185--186). Furthermore, the earmark system is biased towards the rich, since
they have more to spend on lobbying. One particularly displeased president of
Boston University, John Silber, pointed out that ``There are seven national
laboratories that are funded by non-peer review [earmarks] at \$2 billion a
year at MIT, Caltech, Chicago, California, Columbia, Johns Hopkins, and
Harvard.'' (p. 189).

Scientists' dealings with the government are far from the only situations in
which they have abandoned their devotion to honesty and integrity.
Increasingly, scientists have been enticed by the lure of greater financial
gain in corporate work. Nowhere have the ethical failings been more evident
than in overly sensational biomedical industry. A clear example of unethical
misrepresentation is fluconazole.  Fluconazole is an antipsychotic drug
manufactured by Pfizer Inc.  Pfizer, in collusion with unscrupulous academics,
released multiple publications with the same data under different authors to
several scientific journals.This makes it difficult for independent researchers
to prepare, analyze, and compare the drugs from  the different studies via a
process called meta-analysis. Even though there were no hidden side effects,
the presence of multiple instances of the same data skewed the results of
meta-analysis in favor of Pfizer (p. 350).

Scientists hold a unique position in society. They are looked toward by the
public to both find and confirm facts. However, with the presence of
corruption, they repeatedly abuse their standing.  Through the past 50 years,
we have seen acts of unethical behavior ranging from simple greed, such as
abusing grants, to the endangerment of human life, as displayed by Pfizer Inc.
Although actions have been taken to prevent and reduce the occurrences of these
misdeeds, it has not been enough to truly cause a permanent change within the
scientific system. It is up to the individual scientists to be vigilant in
maintaining the true scientific and ethical integrity of the scientific ethos;
it is up to politicians to break down the barrier between national politics and
the scientific establishment in order to hold science accountable to public
interest; and it is up to the citizens to look upon scientific funding with a
critical eye and an intolerance of corruption within the scientific and political arena.

\newpage
\begin{thebibliography}{9}

\bibitem{greenberg}
  Greenberg, Daniel.
  \emph{Science, Money, and Politics: Political Triumph and Ethical Erosion}.
  The University of Chicago Press, Chicago,
  2001.

\end{thebibliography}

\end{document}             % End of document.
